% Template for larmemos -- 13 Jan 99
\documentclass[11pt]{larmemo}
%\pagestyle{secret}             % use this pagestyle for SRD
% Fixed information
% All lines are required.
\divisionname{Applied Computational and\\ 
Theoretical Physics Division}	% center, project, or divison name 
\groupname{X--TM}		% organization number and/or name
\phone{7-5341, 5-5538}		% sender phone, FAX number
\fromms{Mark G. Gray, D409}	% sender initials, mail stop
\originator{mgg}		% memo originator
\typist{mgg}			% memo typist

% Information on this memo
% The \toms, \refno, and \subject commands are required.
\toms{Les Thode, B218}		% recipient initials, mail stop
\refno{X--TM--0113}		% reference number
\subject{Thermonuclear Burn Project:\\
	\LaTeXe\ Style File and
Research Memorandum Reference}  % subject line

% Optional information:
%\thru{nobody, MSnowhere}       % person(s) to send memo through
%\cy{aaa\\bbb}                  % copy list
%\distribution{aaa\\bbb}        % distribution list
%\enc{aaa\\bbb}                 % list of enclosures
%\encas                         % Enc. a/s 
%\attachments{aaa\\bbb}         % list of attachments
%\attachmentas                  % attachment as stated
%\attachmentsas                 % attachments as stated
% Text of the memo

\begin{document}

\maketitle			% make memo header

\section{Summary}
%I want to tell Les that...
I have written a \LaTeXe\ class that templates the research memorandum
format.  The logical structure of this format is described in the book
{\emph Writing Reports to Get Results}.

\section{Background}
Since the major product of my involvement in the TN Burn project is
the production of at least two research memorandum this year, it is
important that I understand the purpose of these reports and have a
good template for their production.  The existing Los Alamos {\tt
memo} class templates have most of the necessary features, but since
they are based on the {\tt letter} format they lack the sectioning
commands suggested in the example research memorandum.  I have created
a new memo class, {\tt larmemo}, which is enhances the standard {\tt
article} class with {\tt t2memo} header fields.

The logical content of the sections is specified by the
aforementioned book, while the physical content is, of course,
specified by the research topic.

\newpage

\section{Investigation}


\section{Outcome}


\end{document}
