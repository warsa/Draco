%%---------------------------------------------------------------------------%%
%% draco-baltab.tex
%% John McGhee
%% $Id$
%%---------------------------------------------------------------------------%%
\documentclass[reqno]{lanl}
\usepackage{subfigure,epsfig,graphicx}
\usepackage{cite}
\usepackage[centertags]{amsmath}
\usepackage{amssymb}
\usepackage[mathcal]{euscript}
\usepackage{tmath}
%\usepackage{doublespace}
%\usepackage[nomarkers]{endfloat}

%%---------------------------------------------------------------------------%%
%% BEGIN DOCUMENT
%%---------------------------------------------------------------------------%%

\begin{document}

%%---------------------------------------------------------------------------%%
%% HEADER STUFF
%%---------------------------------------------------------------------------%%

\title{Draco Balance Table Preliminary Requirements}
\author{John McGhee}
\address{X--TM, MS D409, Los Alamos National Laboratory, Los Alamos, NM
  87544}
\email{mcghee@lanl.gov}

\subjclass{Journal}
\date{\today}

\begin{abstract}
This article serves as a preliminary requirements
document for the Draco balance table utility.
\end{abstract}

\keywords{balance table, draco}
\maketitle

%%---------------------------------------------------------------------------%%
%% BODY
%%---------------------------------------------------------------------------%%

\section*{Introduction}
In general, a balance table
provides a means to confirm that certain terms in some client
balance equation are in fact equal. 
Balance table terms are typically  calculated
      using functions of reaction rate and flows, ie:
      \begin{equation}
       \int \sigma \psi(\Omega) d\Omega \; dE \; dV
      \end{equation}
or,
      \begin{equation}
      \int \Omega \cdot \hat{n} \psi(\Omega) d\Omega \; dE \; dA
      \end{equation}
although other functions are sometimes employed.
      The actual implementation of these
      functions can vary widely from one client to another.
Equation terms are often classified as either
sources ($q$) or sinks ($r$), and a relative
balance is then calculated as $\left| 1-q/r \right|$.
Some of the services and features that
one might typically expect to find in a radiation 
transport balance table are listed
below:

\begin{itemize}
\item Balance for single or multi-group problems, displaying 
       within group details as well as  group totals.
\item Balance for various angular moments - e.g, P$_0$, P$_1$, or
      individual discrete ordinates.
\item Balance computed using a variety of spatial data
      centerings and associated discrete integration rules.
\item Balance for a variety of particle types and associated equations - 
       e.g. material, momentum, radiation, photons, neutrons, 
      charged-particles, electrons and ions.
\end{itemize}

\section*{Objective}
The balance table objective is to provide a general balance table utility
for Draco that will be useful to a wide variety of clients.


\section*{Analysis}

In order to
meet the objective to service a wide variety of clients, two major
features appear to be required. First,  it
appears that most of the
functionality required to actually calculate a balance table entry
(i.e. reaction rates and flows) must remain in the client.
This provides the flexibility to handle arbitrary centerings and integration
methods.
Second, it appears necessary that the client must be
responsible for choosing what values to supply the balance table
and how to label them. This will provides the 
flexibility required to construct balance tables with arbitrary
entries pertinent to a wide variety of balance equations. 
As a result of these two features, the balance table  will not have much of a 
internal computational capability. It will fundamentally be a 
``pretty-print'' utility. 

It should be noted that this analysis places 
the responsibility for constructing a useful balance table
with the client.
The concept of balance is a somewhat arbitrary. For example, the
balance can always be improved by adding some arbitrarily large
column to both source
and sink. For meaningful results entries should be related
to actual values of identifiable terms in client equations.
As a default, templates for certain common equations can be
supplied.


\section*{Requirements}
A loose set of requirements is proposed below in a design-by-contract
format.
\begin{itemize}

\item Balance Table (baltab) obligations
\begin{itemize}
\item Any number of independent balance tables can be
      created in the same problem.
\item Baltab will logically consist of a 2D matrix-like construct with
      rows and columns.  Balance will be computed in the row direction.
\item Baltab may contain any number of rows and columns.
\item Baltab columns will be identified as either source or sink.
\item Baltab will have methods to load data entries individually,
      by row, by column, or en mass.
\item Baltab will provide column sums, if more than one row is present.
\item Baltab will provide source and sink row sums.
\item Baltab will have an option to normalize output to some client
      specified total.
\item Baltab will calculate balance (1-src/snk).
\item Baltab will have display option to print a single row only.
\item Baltab will have display option to print normalization constant.
\item Baltab will have display option to print total net balance only.
\item Baltab will have display option to print prob total row only.
\item Baltab will have display option to print entire table.
\item Baltab will have monitor that prints warning if bal exceeds 
      user input limit.
\item Baltab will have display option to print a ``legend'' or key which
      ``defines'' row column headers using their respective
       clear-text descriptions.
\item Baltab will have several default templates (neutrons, photons,
      charged particles, material, electron, ion) and a capability
      to customize those templates.  Example templates are available
      from the DANTE project.
\item Baltab templates will have a time dependent option.
\end{itemize}

\item Client obligations
\begin{itemize}
\item Client will provide methods for producing entries
      in the balance table and will load baltab entries.
\item Client will identify each column as sink or source.
\item Client will provide baltab title and sub-title.
\item Client will provide short (10 char?) headers for each row and column.
\item Client will provide a clear-text description of each row/column header
      for use in construction of a key or legend.
\item Client will provide a balance limit for printing warning
      messages.
\item Client will provide a normalization constant.
\end{itemize}


\end{itemize}



\section*{A Balance Table Class}

To help flesh out the balance table concept, a sketch
for one possible balance table implementation is presented
below.

\subsection*{Data} The following data items might be present
in a ``balance table'' object.
\begin{verbatim}
string title 
string sub_title
int number_of_columns
int number_of_rows
list<string> column_headers
list<string> row_headers
list<string> column_header_key
list<string> row_header_key
list<list<double>> data_matrix
list<double> row_sums
list<double> column_sums
list<double> balance
double normalization_constant
bool normalize_flag
bool consistency_flag
\end{verbatim}

\subsection*{Methods}

There will be several ``setup'' methods, which would typically
be called once. These include methods to 
 construct, load a template, add a column,
delete a column, and destruct. Other methods that might  be
called repeatedly, say at the end of a time
cycle, will be required. These will include
methods to load data and an update method that calculates row and column
sums, normalization and balance. 
Care must be taken that the sink total is non-zero or a 
divide by zero error may occur during calculations.
Whenever a data item is entered, it should set a ``consistency
flag'' that will be reset when new balance is calculated.
There should also be a clear method that sets all data entries to zero.
Finally there will be a variety of print methods to meet the print requirements
outlined above. If the consistency flag is not set, then a print method
should call the update method prior to printing.

\section{Closing}

This document is under construction. It is presently serving
as a means to simply put a variety of ideas and concepts 
down on paper. Subsequent reorganization
and refinement is expected.

%%---------------------------------------------------------------------------%%
%% BIBLIOGRAPHY
%%---------------------------------------------------------------------------%%

%%\bibliographystyle{aip}
%%\bibliography{../bib/draco}

\end{document}

%%---------------------------------------------------------------------------%%
%% end of draco-baltab.tex
%%---------------------------------------------------------------------------%%


