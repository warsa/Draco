%%---------------------------------------------------------------------------%%
%% baltab-design.tex
%% John McGhee
%% Time-stamp: <99/10/01 17:43:09 mcghee>
%%---------------------------------------------------------------------------%%
\documentclass[reqno]{lanl}
\usepackage{subfigure,epsfig,graphicx}
\usepackage{cite}
\usepackage[centertags]{amsmath}
\usepackage{amssymb}
\usepackage[mathcal]{euscript}
\usepackage{tmath}
%\usepackage{doublespace}
%\usepackage[nomarkers]{endfloat}

%%---------------------------------------------------------------------------%%
%% BEGIN DOCUMENT
%%---------------------------------------------------------------------------%%

\begin{document}

%%---------------------------------------------------------------------------%%
%% HEADER STUFF
%%---------------------------------------------------------------------------%%

\title{Design of the Draco Balance Table Utility}
\author{J.M. McGhee}
\address{X--6, MS D409, Los Alamos National Laboratory, Los Alamos, NM
  87544}
\email{mcghee@lanl.gov}

\subjclass{Journal}
\date{\today}

\begin{abstract}
This article describes the design of the Draco balance table utility.
\end{abstract}

\keywords{draco, balance table, design}
\maketitle

%%---------------------------------------------------------------------------%%
%% BODY
%%---------------------------------------------------------------------------%%

\section*{Overview}

The design will be based on the ``Observer'' pattern found
in the well known book ``Design Patterns'' by Erich Gamma, et al\cite{ga95}.
The observer pattern introduces two classes: an observer and a subject.
The balance table itself fulfills the role of the subject. Different views of
the data contained in the subject are provided by one or more
observer classes.

The subject class will resemble  a row-column spreadsheet like
entity. Each spreadsheet ``cell'' will have a unique row and column
title and will have a method associated with it. This method will
require certain services from the host package. These services
will be provided by the host package, likely through the use of
a traits class.

We envision a hierarchy of balance tables, based on particle type.
Possible derived types include:
\begin{itemize}
\item Neutron Energy Balance
\item Charged Particle Energy Balance
\item Photon Energy Balance
\item Material (electron) Energy Balance
\item Material (ion) Energy Balance
\end{itemize}
each with an option for multi-group and time-dependent.
If a host requires a balance table not
already available, then it should be relatively easy to create a new
derived type.

The possibility of multiple balance tables within a single host package
calls out for a container class in which a number of balance table
subjects can be registered. Each of these three classes is discussed
in detail in the following sections.

\section*{Subject Class}
\section*{Observer Class}
\section*{Container Class}


%%---------------------------------------------------------------------------%%
%% BIBLIOGRAPHY
%%---------------------------------------------------------------------------%%

\bibliographystyle{aip}
\bibliography{../../bib/draco}

\end{document}

%%---------------------------------------------------------------------------%%
%% end of baltab-design.tex
%%---------------------------------------------------------------------------%%


