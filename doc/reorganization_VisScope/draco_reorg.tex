%%---------------------------------------------------------------------------%%
%% draco_reorg.tex
%% Mike Buksas
%% Time-stamp: <04/04/30 08:40:48 tme>
%%---------------------------------------------------------------------------%%
%%
%% Project Vision Statement
%% ------------------------
%% A vision statement explains the project in terms of an end result.
%% It clarifies where the project is going by answering the questions
%%      -What?
%%      -Why?
%%      -How will we know the project is a success?
%%
%% Project Scope Statement
%% -----------------------
%% A scope statement describes how a project will achieve its end
%% result given limited resources.  It sets a project's boundaries by
%% defining the balance between resources, constraints, and
%% deliverables. 
%%
%% Critical Success Factors
%% ------------------------
%% What has to happen for success?  What cannot happen?
%%
%% Risk Assessment
%% ---------------
%% What is the probability of the occurence of each identified risk?
%% What is the impact if it happens?  
%%
%%
%%      ---used with permission and adapted from David A. Schmaltz's 
%%         ``Mastering Projects Workshop: Participant Guide,'' 
%%         True North pgs, Inc., P.O. Box 1532,  Walla Walla, WA 99362, 
%%         projectcommunity.com, Rev. 6, August 2001, copyright.
%%
%%
%% Project Tracking
%% ----------------
%% Track actual results and performance with plans. 
%% 
%%    -Did you change the vision or scope of your project?
%%    -How well did you identify critical success factors?
%%    -What risks did your project realize? ...
%%    -How well were you able to estimate delivery dates?
%%    -What corrective action was taken?
%%    -What lessons have you learned?
%%  
%% The addition of project tracking to the vision and scope statement
%% makes it a living document that is revisited at the completion of
%% the project and whenever risks are realized or vision and scope
%% change. 
%%
%%      ---project tracking section adapted from a LANL ASC Internal 
%%         Assessment of the Jayenne Code Project (CCS-4) by Vicki
%%         Clark and Barbara Hoffbauer (CCN-12).  July 22, 2003. 
%%
%%
%%---------------------------------------------------------------------------%%
\documentclass[note]{ResearchNote_pdf}
\usepackage[centertags]{amsmath}
\usepackage{amssymb,amsthm,graphicx}
\usepackage[mathcal]{euscript}
\usepackage{tmadd,tmath}
%\usepackage{cite}
\usepackage{c++}

%%---------------------------------------------------------------------------%%
%% DEFINE SPECIFIC ENVIRONMENTS HERE
%%---------------------------------------------------------------------------%%
%\newcommand{\elfit}{\ensuremath{\operatorname{Im}(-1/\epsilon(\vq,\omega)}}
%\msection{}-->section commands
%\tradem{}  -->add TM subscript to entry
%\ucatm{}   -->add trademark footnote about entry

%%---------------------------------------------------------------------------%%
%% BEGIN DOCUMENT
%%---------------------------------------------------------------------------%%
\begin{document}

%%---------------------------------------------------------------------------%%
%% OPTIONS FOR NOTE
%%---------------------------------------------------------------------------%%

\toms{Distribution}
\refno{CCS-4:04-21(U)}
\subject{Vision and Scope Statements for Project Thuban}

%-------NO CHANGES
\divisionname{Computer and Computational Sciences}
\groupname{CCS-4:Transport Methods Group}
\fromms{Mike Buksas/CCS-4, MS D409\\
  Tom Evans/CCS-4, MS D409}
\phone{(505)667--7580}
\originator{fml}
\typist{fml}
\date{04/30/2004}
\revisionnum{3}

%-------NO CHANGES

%%---------------------------------------------------------------------------%%
%% DISTRIBUTION LIST
%%---------------------------------------------------------------------------%%

\distribution {
  M.W. Buksas, CCS-4, D-409 \\
  J.D. Densmore, CCS-4, D-409 \\
  T.M. Evans, CCS-4, D-409 \\
  K.G. Thompson, CCS-4, D-409 \\
  T.J. Urbatsch, CCS-4, D-409 \\
  CCS-4 Files.
}

%%---------------------------------------------------------------------------%%
%% BEGIN VISION STATEMENT
%%---------------------------------------------------------------------------%%

\opening

\section*{Vision Statement}

Project Thuban\footnote{Thuban, also known as {\em Alpha Draconis}, is
  a fourth-magnitude star in the constellation Draco. It is notable
  for being the star closest to the north pole from approximately 3000
  BC to 1900 BC, due to the procession of the Earth's axis.}  will add
new components to {\sf Draco}, transforming it into a central
repository for the common \C++ software components, the build system,
the documentation system, (including autodoc and LaTeX document
templates), and bibliography files.

We are undertaking this project to improve access to all of the
affected components by improving and unifying their organization. For
example, there is little sharing of bibliography information because
bibliography files tend to be housed with documents that use them in
the scattered {\tt archive} respositoy. This has led to duplication of
bibliographic entries.

Some observable criteria for success:

\begin{itemize}
\item Increased re-use of bibliographic information with less
  duplication.
\item Increased re-use of contrubuted LaTeX styles and classes.
\item Greater group-wide involvement in contributions to LaTeX
  extensions and envrionment improvements.
\end{itemize}

\section*{Scope Statement}

%%
%% Include those of the following elements that apply to your project
%%

\subsection*{Product of the Project}

The product of project Thuban is a more useable {\sf Draco}
respository.  {\sf Draco} should be more useful for developers who
actively participate in it's evolution and users, who are simply
interested in it's services.  To this end, we propose the following
products:

\begin{itemize}
  \item A central, organized, bibliography repository within {\sf Draco} in
    which it is easy to find references and easy to determine where
    new references should be added.
  \item Externally visible images of the parts of {\sf Draco} repository
    which are of interest to users. E.g. The elisp components for
    xemacs, the document templates, the bibliographies, etc.. should
    have stable locations under {\tt /codes/radtran/environment}.
    Users can be directed to set environment variables (e.g. {\tt
      \$TEXINPUTS, \$BIBINPUTS}) and other configurations (e.g. {\tt
      .xemacs/init.el}) to these stable values.
  \item {\bf NEW} An auto-documentation component of the build system
    which supports seperate doxygen builds of project sub-components.
  \item {\bf NEW} A \LaTeX\ document template library and tools which
    support the automatic generation of both postscript and pdf
    versions of a document without re-editing and without document
    specific-makefiles.
\end{itemize}

\section*{Critical Success Factors}

The goal of this project is improved organization, and from that,
increased productivity. Either of these are diffucult to measure
quantatively, although we can rely on a shared notion of
``organizational esthetics'' to determine whreter the new {\sf Draco}
is better organized and more useful than the old {\sf Draco} + {\sf
  archive}.

So, in order to succeed, this project must be based on a common vision
of what constitutes better organization. Attaining this shared vision,
will require reaching a concensus and adjusting the scope of this
project accordingly. 

\begin{table}[ht]
  \begin{center}
    \caption{Critical success factors for Project Thuban.}
    \label{tab:critical-success}
    \begin{tabular}{|p{4.5cm}|c|c|p{4.5cm}|} 
    \hline
    Factor             &  Character   & Strategy & Comments \\ 
    \hline\hline
    Must attain shared vision of ``organization'' & 
    Dilemma & Change Vision/Scope &
    Discuss alternatives, ask users, compromise and remember that we
    can always change it later. \\ \hline
    \end{tabular}
  \end{center}
\end{table}

\section*{Risk Assessment}

See Table~\ref{tab:risk}.

\begin{table}[ht]
  \begin{center}
    \caption{Risk Assessment for the Project Thuban.}
    \label{tab:risk}
    \begin{tabular}{|p{4.5cm}|c|c|c|p{4.5cm}|} 
    \hline
    Risk & Likelihood & Impact & Importance & Contingency \\
    \hline\hline
    Reorg fails to deliver adequate benefit for time invested. &
    2 & 4 & 8 & 
    Scale back on optional activities. Get back to  work. \\ \hline
                                %
    Draco repository becomes too large to find things
    easily. Confusion over it's multiple parts arises. & 
    5 & 5 & 25 &
    Consider splitting into sub-repositories or completely seperate
    repositories. \\ \hline
    \end{tabular}
  \end{center}
\end{table}

\section*{Project Planning}

Because this project is small in scope, we include the planning
details here, rather then in a seperate document.

\subsubsection*{Implementation}

\begin{itemize}
\item Do not begin until Project CLUBIMC is completed and the current
  Draco repository is stable.

\item Create an {\tt environment} repository in Draco.

\item Move the {\tt draco/elisp/} to {\tt draco/environment/elisp}

\item Move {\tt draco/templates/} to {\tt draco/environment/templates}
  
\item Move the latex classes, latex style and bibliography stlye
  classes from {\tt/archive/tex/} into {\tt/draco/environment/latex/}.
  
\item Gather the bibliography files from {\tt draco/doc} and various
  subdirectories of {\tt archive} into \newline {\tt
    draco/environment/bibfiles/}. Consolidate document-specific
  bibliographies into the topical bibliographies. Fix any documents
  which break.
  
\item Create externally visible copies of the repositories by
  mirroring {\tt draco/environment/} as {\tt
    /codes/radtran/environemnt}.
  
\item {\bf NEW} Re-implement the autodoc part of the build system.
  Allow sub-packages to declare their own {\tt autodoc} directories
  for seperate generation of documentation.

\item {\bf NEW} Modify the \LaTeX\ document template classes to attain
  postscript and pdf independence. The goal is to be able to run latex
  or pdf latex on a document without modification.

\item {\bf NEW} Incorporate {\sf latexmake} into draco. This is a tool
  for automatically building \LaTeX\ documents. 
  
\end{itemize}

\subsubsection*{Proposed directory structure}

\begin{center}
  \ttfamily
  \framebox{
    \parbox{5in}{
      \begin{tabbing}
        \hspace*{0.5cm}\=\hspace{0.5cm}\=\hspace{0.5cm}\=\hspace{0.5cm}\=\kill
        draco/ \\
        \> doc/ ({\rmfamily draco-specific documentation)} \\
        \> src/ \\
        \> \> ds++/ \\
        \> \> \> doc/ {(\rmfamily package-specific documentation
          e.g. release notes)} \\  
        \> \> c4/ \\
        \> \> \> doc/ \\
        \> \> {\rmfamily etc...} \\
        \> environment/ \\
        \> \> templates/ \\
        \> \> elisp/ \\
        \> \> latex/ \\
        \> \> bibfiles/
      \end{tabbing}
    }
  }
\end{center}


\subsubsection*{Notes}

\begin{itemize}
  
\item Package-specific documentation is still bundled with the
  package, e.g. {\tt draco/doc/} for documents about Draco as a
  whole and {\tt draco/src/<package>/doc/} for documentation about
  specific packages.
  
\item The bibliography files should be defined along topics e.g. {\em
    imc}, {\em mc}, {\em comp-sci}, etc...

\item The external copy of the environment information can be created
  in two ways. The repositiory can be checked out at the desired
  location, effectively making it a development location as well.
  Alternatively, the directory can be ``installed'' to this location
  as part of the build system. This means that other clients of draco
  with local installations would be able to specify another location.
  
\item Documents within the Draco doc directories can reference the
  bibliographies via a relative path which points to {\tt
    draco/environment/bibfiles}. Documents elsewhere can either depend
  on having {\tt \$BIBINPUTS} set correctly or use the absolute path
  at \newline {\tt /codes/radtran/environment/bibfiles}. Using {\tt
    \$BIBINPUTS} is more robust. It will be necessary for users to set
  {\tt \$TEXIPUTS} and {\tt \$BSTINPUTS} to use the \LaTeX extensions
  provided in \newline {\tt/codes/radtran/environment/latex/} also.

\end{itemize}
  


\section*{Project Tracking}

%Track actual results and performance with plans.
%
%   Did you change the vision or scope of your project? 
%   
%   How well did you identify critical success factors?  What lessons
%   have you learned?
% 
%   What risks did your project realize?  How well did you estimate the
%   likelihood and impact of the realized risks?  Did you encounter
%   unforseen risks?  What lessons have you learned?
% 
%   How well were you able to estimate delivery dates?  What lessons
%   have you learned?
%
%   What corrective action was taken?
%   
%   How will you improve the vision and scope statements for future
%   projects?

\subsection*{Change log:}

\begin{description}
\item[3/25/04] Submitted initial version for review.
  
\item[3/35/04] Had review session with Tom. We agreed that the
  documents themselves (as opposed to package-specific
  ``documentation'') should be moved to Jayenne. This avoids
  mission-creeping Draco into something bigger. 

  The bibliograpy files are to be treated as part of the common
  envrionment, and so remain behind in draco as part of the
  {\tt environment} directory.  I've added some more detail on the
  external environment repository and how documents should use it.

\item [3/30/04] Added some new items for consideration. Upped version
  number to 2.
  
\item [4/30/2004] Updated Draco with the Thuban directory structure as
  part of the draco-5\_0\_0 release.  This closes out the Thuban
  project.

\end{description}

\closing
\end{document}

%%---------------------------------------------------------------------------%%
%% end of draco_reorg.tex
%%---------------------------------------------------------------------------%%

