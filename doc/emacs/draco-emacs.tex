%%---------------------------------------------------------------------------%%
%% draco-emacs.tex
%% Thomas M. Evans
%% $Id$
%%---------------------------------------------------------------------------%%
\documentclass[11pt]{nmemo}
\usepackage[centertags]{amsmath}
\usepackage{amssymb,amsthm,graphicx}
\usepackage[mathcal]{euscript}       
\usepackage{tabularx}
\usepackage{tmadd,tmath}
\usepackage{cite}
\usepackage{c++}
\usepackage{fancycodes}  % needed to get ``LaTeX'' symbol.
\usepackage{float} % used for C++ example code.
\usepackage{shading}
\usepackage{fancybox}
\usepackage{url}
\usepackage[hang,small]{caption2}
\usepackage{times} % better fonts for PDF

%%---------------------------------------------------------------------------%%
%% DEFINE SPECIFIC ENVIRONMENTS HERE
%%---------------------------------------------------------------------------%%
%\newcommand{\elfit}{\ensuremath{\operatorname{Im}(-1/\epsilon(\vq,\omega)}}
%\msection{}-->section commands
%\tradem{}  -->add TM subscript to entry
%\ucatm{}   -->add trademark footnote about entry

\def\UrlFont{\footnotesize\tt}
\newcommand{\comp}[1]{{\normalfont\texttt{\footnotesize{#1}}}}
\newcommand{\draco}{{\normalfont\sffamily Draco}}
\newcommand{\milagro}{{\normalfont\sffamily Milagro}}
\newcommand{\dante}{{\normalfont\sffamily Dante}}
\newcommand{\xemacs}{{\normalfont\sffamily XEmacs}}
\newcolumntype{Y}{>{\centering\arraybackslash}X}
\newcolumntype{Z}{>{\arraybackslash}X}
\newcolumntype{L}{>{\ttfamily\footnotesize}X}

% how much of the page can be occupied by text when a float is on the
% same page.
\renewcommand{\floatpagefraction}{0.8}
\renewcommand{\textfraction}{0.15}
\renewcommand{\bottomfraction}{0.4}
\renewcommand{\topfraction}{0.9}

\floatstyle{plain}
\setlength{\belowcaptionskip}{3ex plus 4pt minus 2pt}
\setlength{\abovecaptionskip}{2ex plus 4pt minus 2pt}

\newfloat{cxxSampleCode}{!ht}{lom}
  \floatname{cxxSampleCode}{Code Example}

\newenvironment{codeExample}
{\footnotesize 
  \VerbatimEnvironment
  \begin{SaveVerbatim}{\mycode}}%
  {\end{SaveVerbatim}%
  \noindent%
  \parashade[.950]{sharpcorners}{\gdef\outlineboxwidth{.5}%
    \UseVerbatim{\mycode}}\normalsize}

\newenvironment{code}
{\footnotesize 
  \VerbatimEnvironment
  \begin{SaveVerbatim}{\mycode}}%
  {\end{SaveVerbatim}%
  \noindent%
  \UseVerbatim{\mycode}\normalsize}
  

%%---------------------------------------------------------------------------%%
%% BEGIN DOCUMENT
%%---------------------------------------------------------------------------%%
\begin{document}

%%---------------------------------------------------------------------------%%
%% OPTIONS FOR NOTE
%%---------------------------------------------------------------------------%%

\toms{Distribution}
%\toms{Joe Sixpak/XTM, MS B226}
\refno{CCS-4:03--50(U)}
\subject{XEmacs Development Environment for Draco (Revision 2)}

%-------NO CHANGES
\divisionname{Computer and Computational Sciences Division}
\groupname{CCS--4:Transport Methods Group}
\fromms{Kelly G. T\def\UrlFont{\tt}hompson/CCS--4, MS D409\\
Tom M. Evans/CCS--4, MS D409}
\phone{(505)665--3929}
\originator{kgt}
\typist{kgt}
\date{\today}
%-------NO CHANGES

%-------OPTIONS
%\reference{NPB Star Reimbursable Project}
%\thru{P. D. Soran, XTM, MS B226}
%\enc{list}      
%\attachments{list}
%\cy{list}
%\encas
%\attachmentas
%\attachmentsas 
%-------OPTIONS

%%---------------------------------------------------------------------------%%
%% DISTRIBUTION LIST
%%---------------------------------------------------------------------------%%

\distribution {
  CCS-4 MS D409:\\ 
  T.B. Adams, CCS--4, MS409\\
  R.S. Baker, CCS--4, MS409\\
  M.W. Buksas, CCS--4, MS D409\\
  B.A. Clark, CCS--4, MS D409\\
  J.A. Dahl, CCS--4, MS D409\\
  T.M. Evans, CCS--4, MS D409\\ 
  M.G. Gray, CCS--4, MS D409\\ 
  H.G. Hughes, CCS--4, MS D409\\ 
  J.M. McGhee, CCS--4, MS D409\\ 
  J.E. Morel, CCS--4, MS D409\\ 
  G.L. Olson, CCS--4, MS D409\\ 
  S.D. Pautz, CCS--4, MS D409\\ 
  K.G. Thompson, CCS--4, MS D409\\
  S.A. Turner, CCS--4, MS D409\\ 
  T.J. Urbatsch, CCS--4, MS D409\\ 
  J.S. Warsa, CCS--4, MS D409\\
  }

%%---------------------------------------------------------------------------%%
%% BEGIN NOTE
%%---------------------------------------------------------------------------%%

\opening

\section{Comments on Revision 2}
This memo was originally issued by Tom Evans in Sept. of 1999 as
internal memo XTM:99--09(U).  The Draco elisp routines have changed a
bit since then so I am issuing an updated version of Tom's memo.  The
primary changes include the inclusion of an RTT pull-down menu, key
bindings allowing the use mouse buttons and wheel on Linux and a
simplified integration procedure.

\section{Introduction}

\draco~\cite{rn98046} has been built using an informal \xemacs-based
design environment.  As part of the \draco\ reorganization
project~\cite{draco-build} we have sorted through the existing Elisp
functions and macros that have defined this informal \draco\ 
environment.  The files that define the \draco\ development
environment have been cleaned up and released (as of this writing the
release tag was rtt-elisp\_v22).  This release provides a formal
development environment for \draco.

Our goal is for all files in \draco\ to have the same ``look and
feel.''  To accomplish this we employ the \xemacs\ editor with a
defined set of modes and functions that both expedite and homogenize
code and document development in \draco.  Additionally, templates are
provided that contain header information and formats.  Thus, all
source code files in \draco\ will obtain the same ``look'' simply by
using this environment.

The \draco\ development environment consists of a series of Elisp
(\comp{.el}) files that need to be loaded from the user's
\comp{.emacs} file.  The \xemacs\ \draco\ development environment
provides the following features:
\begin{itemize}
\item A set of templates providing standard headers and footers for
  \draco\ source code.
\item A \draco\ \C++ indentation style for \xemacs.
\item Functions to expedite editing of source code (including \LaTeX)
  and package creation in \draco.
\item Pull down \xemacs\ menu for easy access to many \draco\ specific
  commands and programming modes.
\item Specialized key and mouse commands.
\end{itemize}
We will show how to load the \draco\ development environment and
utilize these features in the following sections.

%%---------------------------------------------------------------------------%%

\section{Draco Elisp Files}
\label{chap:elispfiles}

CCS-4 maintains a CVS repository (\comp{elisp}) for files related to
creating a common \xemacs\ development environment.  Included with
these files is a wrapper that allow you to easily incorporate the
common development environment into your own \xemacs\ settings.  This
process is described in section \S\ref{sec:basicelispfiles}.  If you
are not interesting in advanced configurations you only need to read
this one section before continuing on to Chapter
\S\ref{chap:templates}.

However, if you are familiar with Elisp and want to learn about
advanced configuration options or want to contribute to the CCS-4
Elisp archive you should read sections
\S\ref{sec:advancedelispfiles}~-~\S\ref{sec:byte-compile}. 

\subsection{Basic Setup}
\label{sec:basicelispfiles}

This section describes how to customize your personal \xemacs\ 
settings by incorporating the default CCS-4 settings.  This process
requires that you edit the \comp{.emacs} file found in your home
directory so that it loads the CCS-4 \xemacs\ Elisp scripts.  Script
sources, binaries and documentation are published at
\url{/codes/radtran/vendors/xemacs/elisp}.

The simple approach is to copy the code shown in the code example
below into your \comp{.emacs} file.  This code should normally appear
near the top of the \comp{.emacs} file so that your personal
customizations are not overwritten by the CCS-4 default environment.

In this example, only the insertion of the last line is required to
obtain the default CCS-4 \xemacs\ environment.  However, it is
strongly recommended that you create a personal Elisp directory
(usually \url{${HOME}/lib/elisp/} or \url{${HOME}/elisp}) where you
can put your personal customizations. The \xemacs\ name for this
directory will be \comp{my-elisp-dir}.

\begin{codeExample}
;; Basic setup of CCS-4 XEmacs environment.
;;

;; Set variables that point to personal elisp stuff...
(setq my-home-dir (concat (getenv "HOME") "/"))
(setq my-elisp-dir (concat my-home-dir "lib/elisp/"))

;; Add personal elisp directory to the XEmacs load path...
(if (file-accessible-directory-p my-elisp-dir)
   (setq load-path (cons my-elisp-dir load-path)))

;; Load the default CCS-4 XEmacs environment...
(load-library "/codes/radtran/vendors/xemacs/elisp/ccs4defaults")
\end{codeExample}

\subsection{Advanced Setup}
\label{sec:advancedelispfiles}

The files that constitute the \xemacs\ \draco\ development environment
are listed in Table~\ref{tab:elisp}.  These files are published on the
CCS-4 LAN at \url{/codes/radtran/vendors/xemacs/elisp/} or they can be
obtained through a CVS export~\cite{cvs} into a directory of your
choice:

\begin{code}
     $ cvs export -r HEAD elisp
\end{code} % $

Note that not all files listed in the table are required by the
\draco\ development environment.  These additional Elisp components
are included because they may be useful for general work in CCS--4.
In addition to the \url{elisp/} directory, an \url{elisp/templates}
directory is generated from the CVS export.  These templates are used
by functions in \comp{tme-hacks} and are not part of the official
\draco\ development environment.  However, these templates are
currently in use by most CCS-4 developers when writing code and
documents both outside and within \draco.  In general, you should not
need to obtain a copy from CVS since you can simply use the templates
published as a prt of \draco.  

%
\begin{table}[!htbp]%
  \caption{Files that are part of the \draco\ development environment
    distribution, rtt-elisp\_v22.}%
  \label{tab:elisp}
  \begin{center}
    \begin{tabularx}{\linewidth}{
        >{\setlength{\hsize}{0.8\hsize}}Y
        >{\setlength{\hsize}{0.2\hsize}}Y
        >{\setlength{\hsize}{2.0\hsize}}Y}
      \hline\hline
      File & Required & Description \\
    \end{tabularx}
    \begin{tabularx}{\linewidth}{
        >{\setlength{\hsize}{0.8\hsize}}L
        >{\setlength{\hsize}{0.2\hsize}}Y
        >{\setlength{\hsize}{2.0\hsize}}X}
      \hline
      ccs4defaults.el & no & used for default XEmacs setup \\
      Config-key.el & no & describes some key-bindings \\
      Config-pkg.el & yes & does package configurations for
      environments used in (and outside) of \draco \\
      dante-macros.el & no & Dante specific function definitions. \\
      draco-hacks.el & yes & \draco-specific function definitions \\
      draco-rtt.el & yes & loads the \draco\ environment and other
      packages \\
      fl-keywords.el & yes & font-lock keywords used throughout the
      \draco\ package \\
      infer-cc-style.el & yes & functionality for inferring
      \comp{c\--indentation\--styles} \\
      infer-mode.el & yes & functionality for inferring the \xemacs\
      mode from strings in file headers \\
      mppl-mode.el & no & mode definition for MPPL (a Fortran
      preprocessor) \\
      nml-mode.el & yes & \comp{nml} mode definition \\
      pooma-hacks.el & yes & c-style used by the POOMA team \\
      rtt-hacks.el & yes & defines the \comp{rtt-c-style}. \\
      rtt-menu.el & no & creates a pull down menu with many \draco\
      specific functions \\
      tme-hacks.el & yes & contains functions used by the \draco\
      environment \\
      xemacs-setup.el & no & miscellaneous settings \\
      \hline\hline
    \end{tabularx}
  \end{center}
\end{table}

Normally you can use the Elisp publshed at
\url{/codes/radtran/vendors/xemacs/elisp}.  However, if you have a lot
of personal customizations you may need maintain your own copy of the
CCS-4 Elisp scripts.

The \url{elisp/} directory that results from CVS export should be
placed in a suitable location.  In general, one of the following two
locations are recommended:
\begin{code}
     $HOME/elisp
     $HOME/lib/elisp
\end{code}

Note that if the general templates (non-\draco) from \comp{tme-hacks}
are desired then the \url{elisp/} directory should be placed in
\url{$HOME/lib/elisp}.  However, the placement of the elisp %$
directory will not affect the \draco\ development environment
functionality.

In addition to the files found in the CVS repository the Elisp files
published at \url{/codes/radtran/vendors/xemacs/elisp/} have been
byte-compiled to improve \xemacs\ performance (see
section~\S\ref{sec:byte-compile}).  These compiled scripts have a
\comp{.elc} extension.  Also found at this location are some third
party Elisp packages including doxymacs~\cite{doxymacs} and ediff as
well as some info-node pages (C-h i).

\subsection{\comp{.emacs} File}
\label{sec:.emacs}

The \draco\ environment is loaded from the \comp{.emacs} file.
Appendix~\ref{.emacs} lists a model \comp{.emacs} file that loads the
\draco\ development environment.  This file also loads additional
packages such as \comp{dired} and \comp{tme-keys}.  \comp{dired} is an
\xemacs\ mode that displays directories and \comp{tme-keys} primarily
defines personal key-bindings.  Each user can add an Elisp file in
his/her personal elisp directory (i.e.: \url{~/elisp/}) that
re-defines the default CCS-4 key bindings and Elisp \comp{mode-hooks}
that he/she wants.  An example of such a file is \comp{tme-keys} that
may be viewed at

\begin{code}
     /codes/radtran/vendors/xemacs/elisp/tme-keys.el
\end{code}

For convenience, this file is reproduced in \S\ref{tme-keys}.
Additional functionality can be added by users to suit additional
needs outside of \draco\ by simply making their own Elisp files in
this manner.

We will use the \comp{.emacs} file in \S\ref{.emacs} as a template
for what follows.  It is recommended, but not required, that your
personal \comp{.emacs} file to include this block of commands.  The
first part of the \comp{.emacs} file (an extracted section is shown in
the code box below) determines where the user's \$\{HOME\} directory
is located, where the user-installed \comp{elisp/} directory is
located and where the default templates and info directories are
located.  The next block of code loads the CCS-4 Elisp from the
directory specified by the variable \comp{ccs4-elisp-dir}.  This
variable should point to the CVS export location (see
section~\S\ref{sec:advancedelispfiles}).  The last section adds the
user's personal Elisp directory to the default path that \xemacs\ uses
to locate packages.  Prepending the personal Elisp directory is done
after load the CCS-4 packages so that the setting and/or files in the
personal directory override the defaults setup by the CCS-4 scripts.

\begin{codeExample}
;; Basic setup of CCS-4 XEmacs environment.
;;

;; Set variables that point to personal elisp stuff...
(setq my-home-dir (concat (getenv "HOME") "/")
(setq my-elisp-dir (concat my-home-dir "lib/elisp/"))
(setq my-templates-dir (concat my-elisp-dir "templates/"))
(setq my-info-dir (concat my-elisp-dir "info/"))

;; Load the default CCS-4 XEmacs environment...
(setq ccs4-elisp-dir "~/lib/elisp")
(if (file-accessible-directory-p ccs4-elisp-dir)
   (setq load-path (cons ccs4-elisp-dir load-path)))
(load-library "ccs4defaults")

;; Add personal elisp directory to the XEmacs load path...
(if (file-accessible-directory-p my-elisp-dir)
   (setq load-path (cons my-elisp-dir load-path)))
\end{codeExample}

Many \xemacs\ settings are defaulted for you by the script
\comp{ccs4defaults.el} and the other CCS-4 Elisp scripts that it loads
on your behalf.  However, you can modify the default behavior by
defining a few variables.  The Elisp example below demonstrates the
syntax used to override the default values for two variables.  In this
particular example the variables tell the \draco\ Elisp routines to
avoid pooma-style formatting, and to install the RTT \xemacs\ 
pull-down menu.  These Elisp variables should be set in your
\comp{.emacs} file before the \comp{ccs4defaults} script is loaded.
Table~\ref{tab:elispvars} provides a more complete list of options
that can be manually set to your customize your environment.

\begin{code}
     (setq want-pooma-style-by-default nil)
     (setq want-rtt-menu               t)
\end{code}

%
\begin{table}[!htbp]%
  \caption{Elisp variables that override the default CCS-4 settings.}%
  \label{tab:elispvars}
  \begin{center}
    \begin{tabularx}{\linewidth}{
        >{\setlength{\hsize}{0.7\hsize}}Y %
        >{\setlength{\hsize}{0.8\hsize}}Y %
        >{\setlength{\hsize}{1.5\hsize}}Y}
      \hline\hline
      Elisp Varible & Default Value & Description \\
      \hline
    \end{tabularx}
    \begin{tabularx}{\linewidth}{
        >{\setlength{\hsize}{0.7\hsize}}Z %
        >{\setlength{\hsize}{0.8\hsize}}L %
        >{\setlength{\hsize}{1.5\hsize}}X}
      ccs4-elisp-dir   & \url{/codes/radtran/vendors/xemacs/elisp} &
      Directory where the CCS-4 elisp scripts reside. \\ 
      my-home-dir      & \url{${HOME}/} & User's home directory. \\ %$
      my-elsip-dir     & \url{${HOME}/lib/elisp/} & User's private %$
      elisp directory. \\
      my-templates-dir & \url{${HOME}/lib/elisp/templates/} & User's %$
      private elisp/templates directory. \\
      my-info-dir      & \url{${HOME}/lib/elisp/info/} & User's %$
      private elisp/info directory. \\
      want-pooma-style-by-default & nil & Use POOMA style C++ formatting. \\
      want-rtt-menu    & t & Add the RTT menu to the standard \xemacs\ menu.\\
      want-mppl-mode   & nil & Use CCS-4 defaults for editing MPPL sources. \\
      want-tcl-mode    & nil & Use CCS-4 defaults for editing TCL sources. \\
      want-python-mode & t & Use CCS-4 defaults for editing Python sources. \\
      want-nml-mode & nil & Use CCS-4 defaults for editing NML sources. \\
      want-makefile-mode & t & Use CCS-4 defaults for editing Makefiles. \\
      want-cc-mode & t & Use CCS-4 defaults for editing C/C++ sources. \\
      want-auctex-pkg & t & Use CCS-4 defaults for editing \LaTeX\ sources. \\
      want-f90-mode & t & Use CCS-4 defaults for editing F90 sources. \\
      want-fortran-mode & t & Use CCS-4 defaults for editing FORTRAN sources. \\
      config-pkg-verbose & nil & Make \xemacs\ verbose when loading CCS-4 elisp scripts. \\
      \hline\hline
    \end{tabularx}
  \end{center}
\end{table}

\subsection{Selected commentary concerning the default CCS-4 \xemacs\
  development environment}
\label{sec:selcom}

A brief discussion about some of the default settings established in
the CCS-4 Elisp scripts are presented below.  In general, each feature
can be deactivated/activated by setting a \xemacs\ variable before the
CCS-4 script is loaded.  For example, if the Elisp variable
\comp{want-auctex-pkg} has been set to \comp{t} then the script
\comp{Config-pkg.el} will load the \comp{auctex-mode} that is used for
editing \LaTeX\ source.  Normally, the wrapper script
\comp{ccs4defaults.el} sets the default value of this variable except
when the user's \comp{.emacs} has already set the variable before
either \comp{ccs4defaults.el} or \comp{Config-pkg.el} are loaded.  The
remaining variables discussed in this section work in a similar
fashion.

\subsubsection{Auctex}
\label{sec:auctexmode}

\comp{Auctex} is a useful mode for editing \TeX\ and \LaTeX\ source
files.  This mode provides fontification of keywords, menu items and
customized key bindings for the insertion of \LaTeX\ commands, and
more. 

Variable: \comp{want-auctex-pkg} \\
Default value: \comp{t} (set by \comp{ccs4defaults.el}).

\subsubsection{F90}
\label{sec:f90mode}

F90-mode comments...

Variable: \comp{want-f90-mode} \\
Default value: \comp{t} (set by \comp{ccs4defaults.el}).

\subsubsection{Extra key bindings}
\label{sec:keybindings}

If you use the \comp{ccs4defaults.el} script you will be loading some
keyboard shortcuts % and macros defined 
coded by Tom or Jeff.  The following command in \comp{ccs4defaults.el}
is responsible for loading these settings.

\begin{code}
     (load-library "Config-key")
\end{code}

Table~\ref{tab:keybinding1} provides a list of keybindings defined by
\comp{Config-key}.  The syntax \comp{M-x} means hold down the Meta (or
Alt) key while pressing the letter x.  Similarly, \comp{C-h}
indiciates that you should hold down the Ctrl key while pressing the
letter h.  S- represents the shift key.  More complex commands will
have spacing that tell you to release all keys.  For example
\comp{C-x~C-f} should be read as pressing control and x together,
releasing both and then pressing control and f together.  This
particular command is used to open a new file for editing.  

Each custom key binding represents an \xemacs\ function.  Pressing the
key sequence shown in the first column of Table~\ref{tab:keybinding1}
is equivalent to entering \comp{M-x} and entering the corresponding
\xemacs\ Elisp command shown in column 2.  That is to say, typing
\comp{M-?} is equivalent to typing \comp{M-x help-for-help<ret>}.

%
\begin{table}[!htbp]%
  \caption{Custom Keybindings defined in Config-key.}%
  \label{tab:keybinding1}
  \begin{center}
    \begin{tabularx}{\linewidth}{
        >{\setlength{\hsize}{0.6\hsize}}Y %
        >{\setlength{\hsize}{0.8\hsize}}Y %
        >{\setlength{\hsize}{1.6\hsize}}Y}
      \hline\hline
      Keyboard shortcut & \xemacs\ function & Description \\
      \hline
    \end{tabularx}
    \begin{tabularx}{\linewidth}{
        >{\setlength{\hsize}{0.6\hsize}}Z %
        >{\setlength{\hsize}{0.8\hsize}}L %
        >{\setlength{\hsize}{1.6\hsize}}X}
      M-?    & help-for-help        & Provides a list of \xemacs\ help commands. \\
      C-x ?  & describe-key-briefly & Brief description of function bound to key.\\
      \hline
      Keypad /       & byte-compile-file & Compile Elisp file to .elc.\\
      Keypad *       & start-kbd-macro & \\
      Keypad -       & end-kbd-macro & \\
      Keypad +       & call-last-kbd-macro & \\
      Keypad Enter   & other-window & \\
      Keypad C-Enter & gmf-top-other-window & \\
      Keypad 7       & font-lock-fontify-buffer & Redisplays the
      buffer including fontifications of keywords. \\
      \hline
      Pause/Break & delete-other-windows & \\
      F1          & compile & \\
      F2          & goto-line & \\
      C-S-F2      & what-line & \\
      F3          & previous-line & \\
      F4          & next-error & \\
      F5          & insert-function-doc & \\
      F6          & insert-comment-divider & \\
      F7          & save-and-kill & \\
      F8          & previous-buffer & \\
      C-S-F8      & find-companion file & Switch between .hh and
      .cc for a class. \\
      F9          & new-class & \\
      F10         & new-cc-file & \\
      C-F10       & new-hh-file & \\
      F11         & grep & \\
      F10         & new-cc-file & \\
      \hline
      Mouse B3 & kill-region & \\
      Mouse S-B3 & kill-rectangle & \\
      \hline\hline
    \end{tabularx}
  \end{center}
\end{table}


\subsubsection{Personal \xemacs\ customization}
\label{sec:xemacscustomization}

The example \comp{.emacs} shown in Appendix~\S\ref{.emacs} contains
other \xemacs\ customizations not related to the CCS-4 scripts.  For
descriptions of these and other \xemacs\ commands see the \xemacs\ 
help menu, info page (\comp{C-h i}) or the \xemacs\ web
page~\cite{xemacsweb}.  Another good source of information is the
O'Reilly \& Associates text on \xemacs~\cite{ca91}.


\subsection{Byte Compiling Elisp Files}
\label{sec:byte-compile}

To decrease the time required to load and configure \xemacs\ you may
want to byte compile the Draco Elisp files.  This is an optional
step.  There is no requirement to compile these files since emacs will
evaluate the non-compiled elisp as needed.  Here are the steps to
follow to byte compile your elisp routines:

\begin{itemize}
\item Open your elisp directory in an \xemacs\ window (e.g.: C-x C-f 
$\sim$/lib/elisp).  This will open a directory listed in Dired mode.
\item Select all of the \texttt{.el} files. (\%e el).
\item Request \xemacs\ to compile the selected files (B y)
\end{itemize}

If you have any other custom code you may want to byte compile that
as well.  The scripts found at
\url{/codes/radtran/vendors/xemacs/elisp} have already been
byte-compiled for you.

%%---------------------------------------------------------------------------%%

\section{Draco Templates}
\label{chap:templates}

A set of templates used by \draco\ resides in the
\comp{draco/templates/} directory.  These templates are used by the
functions described in \S\ref{sec:func} to make \comp{.cc},
\comp{.tex}, and a host of other files.  Additional templates are
provided to help setup \draco\ packages quickly and easily.  \draco\ 
functions must be called under the \comp{draco/} (or \draco-like, ie.
\comp{milagro/} or \comp{dantev2/}) directory in order to use the
\draco\ templates.  This functionality is explained in
\S\ref{sec:func}.  The template functions in \comp{tme-hacks} can be
used for work outside of \draco.

%%---------------------------------------------------------------------------%%

\section{Draco \C++ Indentation Style}

We have developed a \C++ indentation style for use in \draco.  The
style is called RTT and can be confirmed by querying the value of the
\comp{c-indentation-style} variable.  Figure~\ref{codeExample:samp1} gives an
example of the RTT indentation style.  
%% \cdFramedFigure{Test}{Example
%%  of the RTT \C++-indentation style.}  
By loading \comp{ccs4defaults} (advanced users will recognize that the
wrapper simply loads \comp{draco-rtt} where the real work is done) in
the \comp{.emacs} file this indentation style is set automatically.
% talk about using the re-indent-buffer command?

We note that there still exists a small bug in the \xemacs\ 
implementation of the \comp{namespace} keyword.  This will nominally
occur in definitions of functions within a \comp{namespace}.  To avoid
this error, we set the namespace indentation to zero.  Thus, the user
will not encounter this \xemacs\ \comp{c++-mode} bug when using the
RTT style.

% add doxygen comments to this example?

% Contents of Test.hh
\begin{cxxSampleCode}
\begin{codeExample}
//---------------------------------------------------------------------------//
// Test of EMACS C++ Indentation Style
//---------------------------------------------------------------------------//

namespace rtt_test {

class Test
{
  private:
    // some data
    int x;
    double y;
    
  public:
    // constructor
    Test();
    
    // functions
    int get_x() { return x; }
    
    // inline functions
    template<class T> inline T do_something(T &);
};

// inline function
template<class T> T Test::do_something(T &x)
{
    double sum;
    for (int i = 0; i < x; i++)
    {
        x[i] += y;
        sum += x[i];
    }
    x.push_back(sum);
    
    // return the array-type
    return x;
}

} // end of namespace rtt_test

\end{codeExample}
\caption{Example of the RTT \C++-indentation style.}
\label{codeExample:samp1}
\end{cxxSampleCode}

%%---------------------------------------------------------------------------%%

\section{Draco Functions}
\label{sec:func}

This section describes two families of functions.  The first set,
\S\ref{sec:dfunc}, is provided for use when developing in \draco\ 
only.  These functions must be used under a viable \draco-like system
directory such as \comp{draco/}, \comp{danteV2/}, or \comp{milagro/}.
The second set, \S\ref{sec:gfunc} are general functions that may be
useful to \draco\ developers on projects outside of \draco.  These
function lists are highlights.  There exists additional functions that
have limited, but useful, utility to various users.  Review the source
code for the packages listed in Table~\ref{tab:elisp} for complete
lists and descriptions of all functions.

\subsection{Draco Functions}
\label{sec:dfunc}

The following is a list of functions used exclusively in \draco\.
Most of these commands can be found on the RTT pull-down menu in
\xemacs.
\begin{description}
\item[\comp{draco-package}:] Sets up the required files for a \draco\ 
  package including \comp{configure.in}, \comp{config.h.in},
  \comp{Version.hh}, and \comp{Version.cc}. [RTT menu $\triangleright$
  New Files $\triangleright$ Draco $\triangleright$ New Draco Package]
\item[\comp{draco-class}:] Sets up a \C++ class in \draco; makes a
  \textsl{class}\comp{.hh}, \textsl{class}\comp{.cc}, and a
  \textsl{class}\comp{.t.hh} file. [RTT menu $\triangleright$
  New Files $\triangleright$ New C++ Class]
\item[\comp{draco-cc-head}:] Sets up a \C++ header file in \draco;
  makes a \textsl{header}\comp{.hh} file. [RTT menu $\triangleright$
  New Files $\triangleright$ New C++ Header]
\item[\comp{draco-cc-headin}:] Sets up a \C++ header file in \draco\ 
  that is modified during configuration (see Ref.~\cite{draco-build});
  makes a \textsl{header}\comp{.hh.in} file. [RTT menu $\triangleright$
  New Files $\triangleright$ New C++ Header.in]
\item[\comp{draco-c-head}:] Sets up a C header file in \draco; makes a 
  \textsl{header}\comp{.h} file.
\item[\comp{draco-c-headin}:] Sets up a C header file in \draco\ that
  is modified during configuration (see Ref.~\cite{draco-build});
  makes a \textsl{header}\comp{.h.in} file.
\item[\comp{draco-cc-imp}:] Sets up a \C++ implementation file unit;
  makes a \textsl{base}\comp{.cc} and \textsl{base}\comp{.t.hh} file.
\item[\comp{draco-python}:] Sets up a Python file; makes a
  \textsl{base}\comp{.py} file.
\item[\comp{draco-memo}:] Sets up a LANL-style memo using \LaTeX;
  makes a \textsl{name}\comp{.tex} file.
\item[\comp{draco-note}:] Sets up a LANL-style research note using
  \LaTeX; makes a \textsl{name}\comp{.tex} file.
\item[\comp{draco-article}:] Sets up a RTT-style article using \LaTeX;
  makes a \textsl{name}\comp{.tex} file.
\item[\comp{draco-report}:] Sets up a RTT-style report using \LaTeX;
  makes a \textsl{name}\comp{.tex} file.
\item[\comp{draco-bib}:] Sets up a \LaTeX\ BiB-\TeX\ file in RTT
  format; makes a \textsl{name}\comp{.bib} file.
\end{description}

Remember, these functions must be launched from a directory that is
located under \comp{draco/} or a \draco-like directory.  This is
required because the Elisp functions search up the directory-tree for
a \comp{templates/} directory where package specific template files
are stored.  Thus, if a user is working in \comp{draco/src/c4/} and
calls a \draco\ Elisp function then the directory will be searched
upwards until \comp{draco/templates/} is found.  Therefore, any system
directory-tree that uses the \draco\ model will accept the \draco\ 
Elisp functionality.  However, your package must have a
\comp{templates/} directory populated by the required files listed in
Table~\ref{tab:templatefiles}.

%
\begin{table}[!htbp]%
  \caption{Minimum set of template files required by CCS-4 Elisp}% 
  \label{tab:templatefiles}
  \begin{center}
    \begin{tabularx}{\linewidth}{
        >{\setlength{\hsize}{0.6\hsize}}Y %
        >{\setlength{\hsize}{1.4\hsize}}Y}
      \hline\hline
      Template filename & Description \\
      \hline
    \end{tabularx}
    \begin{tabularx}{\linewidth}{
        >{\setlength{\hsize}{0.6\hsize}}Z %
        >{\setlength{\hsize}{1.4\hsize}}X}
      configure.package.in    & Template for creating package level configure.in. \\
      draco\_art.tex          & Template for creating [?]. \\
      draco\_bib.bib          & Template for creating project level
      \LaTeX\ bibliography file. \\
      draco\_memo.tex         & Template for creating project level
      memo. \\
      draco\_note.tex         & Template for creating project level
      technical note. \\
      draco\_rep.tex          & Template for creating project level
      report. \\
      Makefile.temp           & ? \\
      Makefile.test           & ? \\
      pkg\_Test.cc            & ? \\
      pkg\_Test.hh            & ? \\
      Release.cc              & Template for creating package level
      Release implementation file. \\
      Release.hh              & Template for creating package level
      Release heder file. \\
      template\_c4\_test.cc   & ? \\
      template.cc             & Template for creating package level
      class implementation file. \\
      template.hh             & Template for creating package level
      class header file. \\
      template.py             & Template for creating package level
      python script. \\
      template\_test.cc       & Template for creating package/test
      level unit test driver.\\
      template.t.hh           & Template for creating package level
      class template instantiation file. \\
      \hline\hline
    \end{tabularx}
  \end{center}
\end{table}

The user always has the option of making key-bindings for these
functions.  Key-bindings should be placed in the user's \comp{.emacs}
file or in a file in the \comp{elisp/} directory that is loaded from
\comp{.emacs}.  See \S\ref{tme-keys} for examples.

\subsection{General Functions}
\label{sec:gfunc}

The following is a list of Elisp functions that are useful in
environments that supercede \draco.  This is a highlighted list of
functions that exist in the Elisp distribution described in
Table~\ref{tab:elisp}.  The easiest way to see the functions is to use
the \xemacs\ \comp{help-for-help A} function.  Query the strings
\comp{gmf}, \comp{rtt}, and \comp{tme} to see an almost complete list
of the functions defined in the \comp{elisp/} directory\footnote{Many
  of the interactive \comp{rtt-} prefixed functions are deprecated in
  the new build system.  These have been kept in the Elisp directory
  for reference and for working on previously tagged version of
  \draco.}.  Many of the functions that are prefixed by \comp{tme-}
use the \comp{\$HOME/lib/elisp/templates} directory as the default
location for templates.
\begin{description}
\item[\comp{rtt-find-companion-file}:] Finds the companion to a
  \comp{.hh} or \comp{.cc} file. [F8 or RTT menu $\triangleright$
  Editing RTT Files $\triangleright$ Find companion file]
\item[\comp{rtt-insert-class-doc}:] Inserts a \C++-style class comment 
  block. [RTT menu $\triangleright$
  Editing RTT Files $\triangleright$ Insert comment block]
\item[\comp{rtt-insert-comment-divider}:] Inserts a \C++-style comment 
  divider. [RTT menu $\triangleright$
  Editing RTT Files $\triangleright$ Insert comment divider]
\item[\comp{rtt-insert-function-doc}:] Inserts a \C++-style comment
  block divider. [RTT menu $\triangleright$
  Editing RTT Files $\triangleright$ Insert function comment divider]
\item[\comp{tme-latex-comment-divider}:] Inserts a \LaTeX-style comment
  divider.
\item[\comp{tme-latex-divider}:] Inserts a \LaTeX-style comment block 
  divider.
\item[\comp{tme-latex-radtran-dist}:] Inserts the distribution list of 
  the RTT (\draco) team for LANL memos and research notes.
\item[\comp{tme-latex-xtm-dist}:] Inserts the distribution list of XTM 
  for LANL memos and research notes.
\item[\comp{tme-c-comment-divider}:] Inserts a C-style comment
  divider.
\item[\comp{tme-c-divider}:] Inserts a C-style comment block divider.
\item[\comp{tme-m4-comment-divider}:] Inserts a M4-style comment
  divider.
\item[\comp{tme-m4-divider}:] Inserts a M4-style comment block
  divider.
\item[\comp{tme-makefile-comment-divider}:] Inserts a makefile-style
  (\comp{\#}) comment divider.  This is also useful in \comp{sh-mode},
  \comp{autoconf-mode}, and \comp{python-mode}.
\item[\comp{tme-makefile-divider}:] Inserts a makefile-style
  (\comp{\#}) comment block divider. This is also useful in
  \comp{sh-mode}, \comp{autoconf-mode}, and \comp{python-mode}.
\item[\comp{tme-\textsl{type}-file}:] Makes a template for the
  language \textsl{type}.  These are very similar to the \draco\ 
  functions described in \S\ref{sec:dfunc}; however, these functions
  can be used anywhere, not just under \comp{draco/}.  Additionally,
  the template default location is \comp{\$HOME/lib/elisp/templates}.
\item[\comp{gmf-save-and-kill}:]  Saves a buffer and then kills it.
\end{description}  
A good example of how to incorporate these functions into Elisp
\comp{mode-hooks} is given in the file \comp{tme-keys}; see
\S\ref{tme-keys}.

\section{Other useful tools related to \xemacs\ development}
\label{sec:othertools}

\subsection{PCL-CVS}
\label{sec:pclcvs}

Users may find the \xemacs\ CVS interface to be a helpful tool when
interacting with the CCS-4 CVS repository.  From \xemacs\ you can
start a CVS session by using the RTT menu $\triangleright$ CVS
$\triangleright$ Examine.  \xemacs\ will prompt you for the source
directory and then display a list of files that differ from the files
in the repository.

The PCS-CVS mode also uses Ediff as a drop in replacement of \comp{cvs
  diff}.  This allows you to easily view differences (split pane view
with highlighted text) between the checked out/modified version of a
file an the version in the repository.

\subsection{Doxymacs}
\label{sec:doxymacs}

Doxymacs is an \xemacs\ mode that is useful for highlighting Doxygen
style comments in your source code.

\subsection{Bug, Issue and Feature Tracking}
\label{sec:bugtracking}

CCS-4 also tracks bugs, issues and requested features for the Elisp
archive.  We use the Bugzilla issue tracker found at
\url{https://naxos.lanl.gov/bugzilla}.  Elisp issues should be entered
under the ``Elisp'' component.

%%---------------------------------------------------------------------------%%

\pagebreak
\appendix

\section{Sample \comp{.emacs} File}
\label{.emacs}

This is a sample \comp{.emacs} file that loads the \draco\ development
environment.  Additionally, this file loads user-defined options that
are stored in the \comp{.xemacs-std} file.  The \comp{.xemacs-std}
file is for preferences that the user does not want overwritten.
These options could easily be placed directly in the \comp{.emacs}
file if desired.  See the \xemacs\ \comp{Info} (C-h i) pages for more
help.

{\footnotesize\input{emacs}}

\pagebreak
\section{Sample User Key-Bindings and Mode-Hooks}
\label{tme-keys}

This is a sample key-bindings and \comp{mode-hooks} file.  There are
many ways to achieve this type of functionality in \xemacs.  This file 
is one example.  See the \xemacs\ \comp{Info} pages for more help.

{\footnotesize\input{tme-keys}}

%%---------------------------------------------------------------------------%%

\pagebreak
\bibliographystyle{rnote}
\bibliography{../bib/draco}

\closing 
\end{document}

%%---------------------------------------------------------------------------%%
%% end of draco-emacs.tex
%%---------------------------------------------------------------------------%%
