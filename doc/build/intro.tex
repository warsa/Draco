%%---------------------------------------------------------------------------%%
%% intro.tex
%% Time-stamp: <03/05/12 09:23:35 tme>
%% introduction of Draco build system manual
%%---------------------------------------------------------------------------%%

\section{Introduction}
\label{sec:intro}

The purpose of this document is to give a high-level introduction to
the \draco\ build system (DBS).  A detailed user/developer manual is
processed with \soft{Doxygen}~\cite{doxygen} and is maintained with
the \draco\ source code.  

As a review, we restate the \draco\ mission statement~\cite{rn98046}:
\begin{quote}
  \slshape \draco\ \index{Draco} is a comprehensive, radiation
  transport library that provides highly tested, verified, reusable
  components for serial and parallel, computational physics codes.
\end{quote}
To meet these requirements \draco\ uses rigorous software engineering
concepts including object-oriented/\-generic design, unit testing, and
acyclic (levelized) design principles. The DBS allows \draco\ to
satisfy its mission statement and enforces the concept of acyclic
component design.

The DBS has been in use within CCS--4 (and its predecessor groups)
since January, 1999.  An informal manual, \textit{The Draco Build
  System}, has been used by developers to learn the build system. This
manual rapidly became out-of-date as the DBS was changed and updated.
However, we have now integrated DBS documentation under
\soft{Doxygen}; thus, the in-code \soft{Doxygen} documentation, which
is updated as changes to the DBS are made, has become the official
source of information about the DBS.  HTML versions of the \draco\ 
developer manual, along with PDF versions of the build system manual,
are available on \soft{SourceForge} \index{SourceForge} at
\cmp{http://sf-web.lanl.gov/draco}.

The DBS has been carefully designed.  In particular, we had several
requirements that the build system should satisfy.  These requirements
are:
\begin{itemize}
\item support for simultaneous, multiple configurations;
\item support for \C++ on all ASCI platforms;
\item automated unit and regression testing;
\item the ability to support multiple code projects;
\item support for external vendors;
\item support for multiple languages;
\item extensibility;
\item low-cost on developers to add new packages, code, and tests; 
\item compliance with the GNU coding standard \cite{gnu} \index{GNU
    Coding Standard}.
\end{itemize}
The last requirement, compliance with the GNU Coding Standard, is
important because the \make\ targets within \draco\ conform to a
referenced standard.

Compliance with the GNU Coding Standard implies the use of GNU
autotools \index{GNU autotools}, \autoconf\ and
\automake~\cite{autoconf}, and \make~\cite{gmake}.  We have recently
upgraded the DBS to conform to the latest versions of these tools
(\autoconf\ 2.57).

%%---------------------------------------------------------------------------%%

\subsection{Definitions and Conventions}

Before continuing we shall clarify the terminology and typeface
conventions that will be employed throughout the remainder of this
manual.  The definitions that we use here are for convenience.  They
are not to be interpreted as an ``universal standard.''  They are
simply used to make sure that the concepts illucidated within this
manual have a common point of reference.

A \latin{product} is anything that is produced from a source code
tree~\cite{ja94}.  A \latin{system} is a code, or a group of codes,
that persist over time~\cite{tn98}.  A \latin{project} is an
undertaking that has a definite beginning and ending date, and it
produces a product.  A \latin{package} is one component of a system
(package and components are used interchangeably).  Packages normally
reside in a single directory in the source code tree; although, that
directory may have subdirectories.  However, packages are sometimes
used to refer to larger units.  For example, a code package may be a
system that contains many components.  In this case, package has macro
(system level) and micro (system-component level) connotations.

Table ~\ref{tab:tfaces} show the typefaces that we will employ
\begin{table}
  \begin{center}
    \caption{Typefaces used throughout the text.}
    \label{tab:tfaces}
    \begin{tabular}{c}\hline\hline
      \sys{code systems} (\draco) \\
      \pkg{packages} (\dsxx) \\
      \cmp{files} (\comp{Makefile}) \\
      \vble{variables} (\comp{draco/src/\vble{pkg}/}) \\
      \soft{software programs} (\autoconf) \\ \hline\hline
    \end{tabular}
  \end{center}
\end{table}
throughout the text to better distinguish certain elements.  In
general, anything that exists on a computer screen (directory trees,
files, etc) is typefaced using \cmp{typewriter} font.  Files are
distinguished in the standard UNIX way by appending the following
symbols after the name, \cmp{*} for executables, \cmp{/} for
directories, and @ for links.  Computer screen prompts are represented
by the \cmp{\$} symbol.

%%---------------------------------------------------------------------------%%

\subsection{DBS Support and Procurement}

Questions about procuring a copy of the DBS or its use can be
directed to:
\begin{center}
  \begin{tabular}{llc}\hline\hline
    \multicolumn{1}{c}{Name} & \multicolumn{1}{c}{Email} &
    Group \\ \hline
    Tom Evans & tme@lanl.gov & CCS--4 \\
    Kelly Thompson& kellyt@lanl.gov & CCS--4 \\ \hline\hline
  \end{tabular}
\end{center}
Additional information is available on the \draco\ \soft{SourceForge}
site, \cmp{http://sf-web.lanl.gov/draco}.